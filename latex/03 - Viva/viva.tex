\documentclass[11pt]{beamer}
\usetheme{Pittsburgh}
\usepackage[utf8]{inputenc}
\usepackage[english]{babel}
\usepackage{amsmath}
\usepackage{amsfonts}
\usepackage{amssymb}
\usepackage{graphicx}

\usepackage{parcolumns}
\usepackage{siunitx}

\usepackage{pstricks}
\usepackage{pst-optexp}
\usepackage{xkeyval}

\author{Davide Bazzanella}
\title{Multidimensional Imaging of Ultrafast Laser Pulses}
%\setbeamercovered{transparent} 
\setbeamertemplate{navigation symbols}{} 
\logo{\includegraphics[height=0.8cm]{Imperial_2_Pantones.eps}\vspace{233pt}\hspace{264pt}} 
\institute{Imperial College London} 
\date{16$^{\mathrm{th}}$ March 2016} 
\subject{}

\defbeamertemplate*{title page}{customized}[1][]
{
\bigskip
\bigskip
\begin{center}
	{
	\usebeamerfont{title}\insertinstitute}\par
	\bigskip
	\bigskip
	{BSc 3rd Year Project}\par
	\bigskip
 	{\usebeamerfont{title}\usebeamercolor[fg]{title}\inserttitle}\par
	\bigskip
	\bigskip
	{\usebeamerfont{date}\insertdate}\par
	\bigskip
	\bigskip
	\begin{parcolumns}{3}
		\colchunk[1]{Assessor:\\ \textbf{John Tisch}}
		\colchunk[2]{Supervisor:\\ \textbf{Tobias Witting}}
		\colchunk[3]{Student:\\ \textbf{Davide Bazzanella}}
	\end{parcolumns}
\end{center}
}

%\setbeamertemplate{title page}
%{
%	\vbox{}
%	\vfill
%	\begin{centering}
%		{\usebeamerfont{title}\usebeamercolor[fg]{title}\inserttitle}
%		\vskip0.2em
%		{\usebeamerfont{subtitle}\usebeamercolor[fg]{subtitle}\insertsubtitle}
%		\vskip2em\par
%		\small\insertauthor\par
%		\vskip0.7em\par
%		\tiny\insertdate\vskip1em\par
%	\end{centering}
%% 	\vfill
%}

\begin{document}

\begin{frame}
\titlepage
\end{frame}

\begin{frame}{Outline}
\tableofcontents
\end{frame}

\section{Introduction}
\begin{frame}{Introduction}
\textbf{Aim}:\\
	measure the 3D spatial profile of an ultrafast laser pulse

	\vspace{15pt}
\textbf{Challenge}:\\
	short duration of laser pulses, down to few femtoseconds ($\SI{e-15}{\s}$)
	
	\vspace{15pt}
\textbf{Method}:\\
	direct measurement of single pulse is impossible,
	measure intensity averaged on many pulses
\end{frame}

\begin{frame}

\textbf{Techniques availables}:\\
\begin{itemize}
\item Spectrography
\begin{itemize}
\item Frequency-Resolved Optical Gating (FROG and variations)
\end{itemize}
\item Interferometry
\begin{itemize}
\item Spectral Phase Interferometry for Direct Electric field Reconstruction (SPIDER and variations).
\item Fourier-Transform Interferometry
\end{itemize}
\end{itemize}	
%There are already some developed techniques such as FROG and SPIDER, and their variations (e.g. SEA-SPIDER).

\textbf{Our target}:\\
Combined use of Fourier-Transform Interferometry and SEA-SPIDER.


\end{frame}

\section{Self-Coherence Interferometry}
\begin{frame}{Self-Coherence Interferometry}

\begin{parcolumns}{2}
	\colchunk[1]{
		Studies the interference between a wave and a copy of itself with a certain delay.

%		\vspace{100pt}
		May be achieved with the use of a Mach-Zender interferometer
	}
	\colchunk[2]{
		image
	}
\end{parcolumns}
\end{frame}

%\section{Setup}
%\begin{frame}{Mach-Zender Interferometer}
%\begin{figure}
%	\includegraphics[scale=1]{text6260-83.png}
%\end{figure}
%\end{frame}

\section{Setup}
\begin{frame}{Mach-Zender Interferometer}
\begin{center}
\begin{pspicture}(10,7)
	\pnodes(0,6){IN}(10,3){OUT}
	\pnodes(1,6){BSA}(5.5,3){BSB}
	\pnodes(1,1){C}(2.5,1){D}(2.5,3){E}(4.5,3){Focus}
	\pnodes(8,6){G}(8,4.5){H}(5.5,4.5){I}
	
	\addtopsstyle{OptComp}{mirrorwidth=1.5, mirrortype=extended}
	\addtopsstyle{OptComp}{bssize=1, bsstyle=plate}
	\addtopsstyle{Beam}{fillstyle=solid, fillcolor=red!50!white, linecolor=red, opacity=0.2}
	
	\beamsplitter(IN)(BSA)(C){BS1}
	
	\mirror(BSA)(C)(D){M1}
	\mirror(C)(D)(E){M2}
	\oapmirror[oapmirroraperture=1.5](D)(E)(Focus){OAP}
	\pinhole[outerheight=1,innerheight=0.1,phlinewidth=0.1](Focus)(Focus){PH}
	
	\beamsplitter(I)(BSB)(OUT){BS1}
	
	\mirror(BSA)(G)(H){M3}
	\mirror(G)(H)(I){M4}
	\mirror(H)(I)(BSB){M5}
	
	\drawwidebeam[beamwidth=0.4](IN){1-6}(OUT)
	\drawwidebeam[beamwidth=0.4](IN)(BSA){7-9}{6}(OUT)
	
	\pnode(11,7){Z}
	\optbox[optboxsize=1.6 3.1](7,5.25)(9.2,5.25)
	\rput[r](9.65,5.5){$\tau$\psline[arrows=<->](-0.65, -0.15)(0.35, -0.15)}
	
\end{pspicture}
\end{center}
\end{frame}

\section{Data Analysis}
\begin{frame}{Fourier Transform Spectrometry}
% replace with block diagram
\begin{itemize}
	\item acquire $I(x,y,\tau.)$ at different $\tau.$ delays.
	\item apply Fourier transform to $I(x.,y.,\tau)$ for every pixel $(x.,y.)$, obtaining $\tilde{I}(x.,y.,\omega)$.
	\item isolate the interference pattern (remove constant values).
	\item multiply reference phase (pinhole) as function of $\phi(x,y,\omega)$
	\item apply the inverse Fourier transform to $\tilde{I}(x.,y.,\omega)$, obtaining $I_{pulse}(x.,y.,\tau)$ for every pixel $(x.,y.)$.
\end{itemize}

\end{frame}

\begin{frame}{Fourier Transform Spectrometry}

\begin{pspicture}(4,7)
	\psset{fillstyle=solid, fillcolor=white}
	\optbox[optboxsize=2 1](2.45,6.45)(3.45,6.45)
	\optbox[optboxsize=2 1](2.3,6.3)(3.3,6.3)
	\optbox[optboxsize=2 1](2.15,6.15)(3.15,6.15)
	\optbox[optboxsize=2 1](2,6)(3,6)
	\rput[r](1.4,6){$y$}
	\rput[r](2.6,5.3){$x$}
	\rput[r](4,5.6){$\tau$}
	
	
	\optbox[optboxsize=2 1](2.45,4.45)(3.45,4.45)
	\optbox[optboxsize=2 1](2.3,4.3)(3.3,4.3)
	\optbox[optboxsize=2 1](2.15,4.15)(3.15,4.15)
	\optbox[optboxsize=2 1](2,4)(3,4)
	\rput[r](1.4,4){$y$}
	\rput[r](2.6,3.3){$x$}
	\rput[r](4,3.6){$\omega$}
	
	
	\optbox[optboxsize=2 1](2.3,2.5)(3.3,2.5)

	
	\optbox[optboxsize=2 1](2.45,0.95)(3.45,0.95)
	\optbox[optboxsize=2 1](2.3,0.8)(3.3,0.8)
	\optbox[optboxsize=2 1](2.15,0.65)(3.15,0.65)
	\optbox[optboxsize=2 1](2,0.5)(3,0.5)
	\rput[r](1.4,0){$y$}
	\rput[r](2.6,-0.2){$x$}
	\rput[r](4,0.1){$\tau$}
\end{pspicture}
\end{frame}

\section{Results}
\begin{frame}{Results for the Ti:S laser}

\end{frame}
\section{Conclusion}
\begin{frame}{Conclusion}
\textbf{Improvements}:\\
	\begin{itemize}
		\item Use of corner cube retroreflectors (CCR) for better alignment
		\item Use of piezo stage for better sampling
		\item Use 90/10 beamsplitter in place of BS1
	\end{itemize}
	
	\vspace{5pt}
\textbf{Summary}:\\
	We would have some results if we were given the right tools.
\end{frame}
\begin{frame}
Thanks to the MATLAB customer service because they are there when you need it. They have a support service that the Ihmperihal Colilege can only dream it.
\end{frame}
\end{document}

%%%%%%%%%%%%%%%%%%%%%%%%%%%%%%%%%%%%%

%\documentclass[11pt, leqno]{beamer}
%\usetheme{Dresden}
%\usepackage[utf8]{inputenc}
%\usepackage[english]{babel}
%\usepackage{amsmath}
%\usepackage{amsfonts}
%\usepackage{amssymb}
%\usepackage{graphicx}
%\usepackage[compatibility=false]{caption}
%\usepackage{subcaption}
%\usepackage{multicol}
%
%\usepackage{siunitx}
%\usepackage{wrapfig}
%%\usepackage{float}
%
%\usepackage{caption}
%\usepackage{subcaption}
%
%\usepackage{parcolumns}
%\usepackage{amsthm}
%\usepackage{sidecap}
%\usepackage{physics}
%
%\usepackage{multirow}
%\usepackage{array}
%
%\defbeamertemplate*{title page}{customized}[1][]
%{
%\begin{center}
%  \includegraphics[height=2.3cm]{unitn_logo}\\
%  \usebeamerfont{institute}\insertinstitute\par
%  \bigskip
% {\usebeamerfont{title}\usebeamercolor[fg]{title}\inserttitle}\par
%  \bigskip
%  \bigskip
%  \bigskip
%  {\usebeamerfont{date}}\insertdate\par
%  \bigskip
%  \bigskip
%\begin{parcolumns}{2}
%   \colchunk{Thesis Advisor:\\ \textbf{Lorenzo Pavesi}\\ \textbf{Mattia Mancinelli}}
%   \colchunk[2]{Student:\\ \textbf{Davide Bazzanella}}
%\end{parcolumns}
%\end{center}
%}
%
%\author{Davide Bazzanella}
%\title{Thermal tuning of phase-matching in a multimodal SOI waveguide with $\chi^{(3)}$ non linearity}
%\date{28 settembre 2015} 
%%\textsc{\LARGE \textbf{DEPARTMENT OF PHYSICS}}\\[0.5cm]
%%\textsc{\LARGE \textbf{DEGREE IN PHYSICS – LAUREA IN FISICA}}\\[2.5cm]
%%\textsc{\Large THESIS}\\[1.3cm]
%
%
%
%%\setbeamercovered{transparent} 
%\setbeamertemplate{navigation symbols}{} 
%%\logo{} 
%%\institute{} 
%%\subject{} 
%
%\usepackage{lmodern}
%\newcommand{\frameofframes}{/}
%\newcommand{\setframeofframes}[1]{\renewcommand{\frameofframes}{#1}}
%
%\setframeofframes{of}
%\makeatletter
%\setbeamertemplate{footline}
%  {%
%    \begin{beamercolorbox}[colsep=1.5pt]{upper separation line foot}
%    \end{beamercolorbox}
%    \begin{beamercolorbox}[ht=2.5ex,dp=1.125ex,%
%      leftskip=.3cm,rightskip=.3cm plus1fil]{author in head/foot}%
%      \leavevmode{\usebeamerfont{author in head/foot}\insertshortauthor}%
%      \hfill%
%      {\usebeamerfont{institute in head/foot}\usebeamercolor[fg]{institute in head/foot}\insertshortinstitute}%
%    \end{beamercolorbox}%
%    \begin{beamercolorbox}[ht=2.5ex,dp=1.125ex,%
%      leftskip=.3cm,rightskip=.3cm plus1fil]{title in head/foot}%
%      {\usebeamerfont{title in head/foot}\insertshorttitle}%
%      \hfill%
%      {\usebeamerfont{frame number}\usebeamercolor[fg]{frame number}\insertframenumber~\frameofframes~\inserttotalframenumber}
%    \end{beamercolorbox}%
%    \begin{beamercolorbox}[colsep=1.5pt]{lower separation line foot}
%    \end{beamercolorbox}
%  }
%\makeatother
%
%\begin{document}
%ciao 
%%\frame{\titlepage}
%
%\end{document}